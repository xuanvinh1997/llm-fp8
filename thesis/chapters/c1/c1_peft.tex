\section{Parameter-Efficient Fine-Tuning Methodologies}

\subsection{LoRA: Low-Rank Adaptation}

\subsubsection{Mathematical Foundation}
The intrinsic dimensionality hypothesis suggests weight updates during fine-tuning lie in a low-dimensional subspace.

\subsubsection{Formulation}
For a pre-trained weight matrix $W_0 \in R^{d \times k}$, LoRA parameterizes the update as:
\begin{equation*} W = W_0 + \Delta W = W_0 + B \cdot A
\end{equation*}
Where: $B \in R^{d \times r}, A \in R^{r \times k}, r \ll \min(d,k)$

\subsubsection{Forward Pass}
\begin{equation*} h = W_0 \cdot x + (B \cdot A) \cdot x = W_0 \cdot x + B \cdot (A \cdot x)
\end{equation*}

\subsubsection{Initialization}
\begin{itemize}
    \item A: Random Gaussian initialization
    \item B: Zero initialization ($\Delta W = 0$ at start, preserving pre-trained behavior)
\end{itemize}

\subsubsection{Scaling Factor}
\begin{equation*} h = W_0 \cdot x + (\alpha/r) \cdot B \cdot A \cdot x
\end{equation*}
Where $\alpha$ is a hyperparameter controlling adaptation strength. Setting $\alpha = r$ at the start of experiments allows changing $r$ without re-tuning learning rate.

\subsubsection{Parameter Reduction}
\begin{verbatim}
Full fine-tuning: d x k parameters
LoRA: r x (d + k) parameters

Reduction factor: dk / (r(d+k)) approx d/(2r) for d approx k
Example: d=k=4096, r=8 -> 256x reduction
\end{verbatim}

\subsubsection{Target Modules}
Typically applied to attention projections:
\begin{itemize}
    \item Query ($W_q$): Most impactful
    \item Value ($W_v$): Second most impactful
    \item Key ($W_k$): Less impact
    \item Output ($W_o$): Moderate impact
    \item FFN layers: Useful for some tasks
\end{itemize}

\subsubsection{Merging for Inference}
\begin{equation*} W_{\text{deployed}} = W_0 + (\alpha/r) \cdot B \cdot A
\end{equation*}
No additional inference latency—adapters are absorbed into base weights.

\subsection{QLoRA: Quantized LoRA}

\subsubsection{Three Key Innovations}

\paragraph{1. NF4 Base Model Quantization}
Store frozen base model in 4-bit NF4, dequantize on-the-fly for computation:
\begin{verbatim}
Compute: Y = dequant(W_NF4) * X + B*A*X
         = (scale * nf4_lookup[indices]) * X + B*A*X

Storage: 4 bits/weight + 32-bit scale per group (e.g., 64 weights)
Compute: BF16 precision
\end{verbatim}

\paragraph{2. Double Quantization}
Quantize the quantization constants themselves:
\begin{verbatim}
Standard: 32-bit scale per group of 64 weights -> 32/64 = 0.5 bits/weight overhead
Double quant: 8-bit scale with 256-group second-level quantization
             -> 8/64 + 32/256 = 0.125 + 0.125 = 0.25 bits/weight
Savings: ~0.37 bits/weight (approx 3GB for 65B model)
\end{verbatim}

\paragraph{3. Paged Optimizers}
Handle memory spikes during gradient checkpointing:
\begin{verbatim}
When GPU memory exhausted:
1. CUDA unified memory pages optimizer states to CPU
2. Computation continues with automatic page faults
3. States paged back on access

No code changes required-transparent to training loop
\end{verbatim}

\subsubsection{Memory Breakdown (65B model)}
\begin{tabular}{l|l|l}
\textbf{Component} & \textbf{Full FT} & \textbf{QLoRA} \\
\hline
Base model weights & 130GB & ~16GB (4-bit) \\
Gradients & 130GB & 0 (frozen base) \\
Optimizer states & 260GB & ~0.5GB (LoRA only) \\
Activations & Variable & ~4GB (checkpointing) \\
LoRA adapters & N/A & ~0.5GB \\
\hline
Total & >520GB & ~21GB \\
\end{tabular}
