\chapter{Introduction}
Large Language Models (LLMs) have achieved remarkable capabilities across a range of tasks, from natural language understanding to complex reasoning. However, this prowess comes with steep resource demands that pose significant challenges for both training and deployment.

The exponential growth in large language model parameter counts and context lengths has created unprecedented computational and memory demands. State-of-the-art models often consist of billions of parameters, requiring substantial memory and computational power for training and inference. Traditional mixed-precision approaches using FP16 and BF16 formats, while effective, still impose significant memory overhead that limits model scalability and training throughput. For instance, training a 3 billion parameter model with BF16 precision requires over 82 GB of GPU memory, while an 8 billion parameter model demands similar or higher memory allocation.

The introduction of 8-bit floating-point (FP8) formats offers the potential to halve memory requirements while maintaining numerical stability, but current implementations face challenges in optimally leveraging the distinct characteristics of available FP8 variants. NVIDIA's standardization of two FP8 formats—E4M3 (4-bit exponent, 3-bit mantissa) and E5M2 (5-bit exponent, 2-bit mantissa)—presents a fundamental trade-off between precision and dynamic range. Existing FP8 training approaches predominantly employ uniform format assignment strategies, but our analysis reveals that different transformer components exhibit distinct computational patterns that benefit from different FP8 formats.

% \section{Problem}

\textbf{Low-bit quantization} has emerged as a promising solution to shrink model size and speed up computation, making deployment of LLMs on edge hardware more practical. Quantization involves using reduced numerical precision to represent model weights and activations. By compressing model parameters from the standard 32-bit floating point (FP32) down to 8-bit or lower, we can dramatically cut memory usage and computational load. Recent advances in mixed-precision techniques even allow combining different precisions (e.g. 16-bit and 4-bit) in one matrix multiplication operation to balance speed and accuracy. Hardware vendors and researchers are actively exploring ultra-low precision arithmetic (int8, int4, even binary) to push LLMs onto resource-constrained environments.
\section{Motivation}\label{sec:motivation}

Why are LLMs hard to deploy on edge devices? The fundamental issue is scale. Modern Transformer-based LLMs achieve higher accuracy primarily by scaling up parameter counts and training data. This leads to massive models that strain memory and compute resources. In general, deploying an LLM requires loading all its weights into memory (GPU VRAM or RAM) and performing billions of math operations per inference. Edge devices, however, are constrained in memory (often a few GB) and lack the specialized matrix acceleration of large data-center GPUs.

\subsection{Qwen2.5-1.5B and Efficient Small-Scale LLMs}
Powerful open-source models such as \textbf{Qwen 2.5} deliver impressive general-purpose performance, yet their accuracy deteriorates once the task domain narrows to highly structured mathematics.  Bridging that gap ordinarily requires full-parameter fine-tuning, a process that is both memory-hungry and compute-intensive for billion-parameter networks.  As a result, research laboratories and edge-computing practitioners alike face prohibitive costs when attempting to adapt state-of-the-art language models to specialized workloads.

Our objective is therefore two-fold: \emph{(i)} preserve the strong reasoning capabilities of Qwen 2.5 on domain-specific maths tasks, and \emph{(ii)} compress the computational footprint so the refined model can run on modest hardware budgets.  Recent accelerator generations provide two complementary avenues for achieving this goal.

\textbf{FP8 training.}  
NVIDIA Hopper- and AMD CDNA3-class GPUs expose native 8-bit floating-point (FP8) tensor cores.  Training directly in FP8 slashes memory consumption by up to 75 \% relative to FP32 while simultaneously boosting arithmetic throughput, enabling us to iterate on large models without resorting to multi-node clusters.

\textbf{FP8 post-training quantization.}  
Once fine-tuning converges, we further compress the model by converting all weight tensors from FP16/BF16 to FP8.  This step multiplies the capacity of a given GPU by roughly two, shortens inference latency, and lowers energy draw—without incurring a noticeable loss in perplexity or downstream accuracy.

By combining FP8-aware optimization with post-training quantization, the project demonstrates that \textsc{Qwen 2.5-1.5B} can retain its mathematical reasoning prowess while fitting within the memory envelope of a single H100 NVL or even smaller edge devices, thereby widening access to high-quality, domain-tuned language models.


\section{Dataset}
To train our reasoning model, we compiled a dataset covering many mathematical and logical problems. This dataset includes various problem types such as arithmetic operations, algebraic equations, logical reasoning, and multi-step problem solving to ensure comprehensive coverage of mathematical reasoning capabilities.
\section{Our Approach}\label{sec:approach}

This section presents our systematic approach to layer-wise FP8 format assignment for transformer training. Our methodology consists of three main components: computational pattern analysis, layer-wise format assignment strategy, and implementation architecture.

\subsection{Layer-Wise Format Assignment Strategy}

Based on our analysis of computational patterns in transformer components, we propose the following systematic assignment strategy:

\textbf{MLP Components:}
For all MLP layers $\ell \in \mathcal{L}_{\mathrm{MLP}}$, we assign weights, activations, and gradients to E4M3 format:
\begin{equation}
\forall \ell \in \mathcal{L}_{\mathrm{MLP}}: \quad W_{\ell}, A_{\ell}, G_{\ell} \mapsto \text{E4M3}
\end{equation}

\textbf{Attention Components:}
For attention layers $\ell \in \mathcal{L}_{\mathrm{Attn}}$, we apply mixed format assignment:
\begin{equation}
\forall \ell \in \mathcal{L}_{\mathrm{Attn}}: \quad
\begin{cases}
Q_{\ell}, K_{\ell} \mapsto \text{E5M2} \\
V_{\ell}, O_{\ell} \mapsto \text{E4M3} \\
G_{Q,\ell}, G_{K,\ell} \mapsto \text{E5M2} \\
G_{V,\ell}, G_{O,\ell} \mapsto \text{E4M3}
\end{cases}
\end{equation}

where $W$, $A$, and $G$ denote weights, activations, and gradients respectively, and $Q$, $K$, $V$, $O$ represent the standard attention projections.

This assignment strategy optimizes the precision-range trade-off for each component type:
\begin{itemize}
\item E4M3 for operations requiring high precision within moderate ranges (MLP operations, value projections)
\item E5M2 for operations requiring wide dynamic range coverage (query-key interactions, attention gradients)
\end{itemize}

\subsection{Implementation Architecture}

Our implementation leverages NVIDIA's Transformer Engine to provide seamless integration with existing PyTorch workflows. The architecture involves systematic module replacement where standard PyTorch layers (\texttt{nn.Linear}, \texttt{nn.LayerNorm}, \texttt{RMSNorm}) are replaced with their Transformer Engine equivalents that support format-specific FP8 computation.

The layer-wise format assignment is implemented through a two-stage process:
\begin{enumerate}
\item \textbf{Model Conversion:} Convert pre-trained model layers to Transformer Engine layers while preserving original parameters
\item \textbf{Format Configuration:} Apply layer-specific FP8 format configurations based on component type (MLP vs Attention)
\end{enumerate}

\subsection{Training Configuration}

We validate our approach on two model scales:
\begin{itemize}
\item \textbf{Llama-3.2-3B}: 3 billion parameters, 32 layers, 32 attention heads
\item \textbf{Llama-3.1-8B}: 8 billion parameters, 32 layers, 32 attention heads
\end{itemize}

Models are trained on 100K instruction-response pairs from the OpenMathInstruct-2 corpus for 3 epochs with sequence length of 512 tokens. We use AdamW optimizer with learning rate 1e-5, batch sizes optimized per model scale, and gradient accumulation of 4 steps.

% \input{chapters/c1/c1_application}
